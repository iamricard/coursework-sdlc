\documentclass[english,a4paper,]{report}
\usepackage{lmodern}
\usepackage{amssymb,amsmath}
\usepackage{ifxetex,ifluatex}
\usepackage{fixltx2e} % provides \textsubscript
\ifnum 0\ifxetex 1\fi\ifluatex 1\fi=0 % if pdftex
  \usepackage[T1]{fontenc}
  \usepackage[utf8]{inputenc}
\else % if luatex or xelatex
  \ifxetex
    \usepackage{mathspec}
  \else
    \usepackage{fontspec}
  \fi
  \defaultfontfeatures{Ligatures=TeX,Scale=MatchLowercase}
\fi
% use upquote if available, for straight quotes in verbatim environments
\IfFileExists{upquote.sty}{\usepackage{upquote}}{}
% use microtype if available
\IfFileExists{microtype.sty}{%
\usepackage[]{microtype}
\UseMicrotypeSet[protrusion]{basicmath} % disable protrusion for tt fonts
}{}
\PassOptionsToPackage{hyphens}{url} % url is loaded by hyperref
\usepackage[unicode=true]{hyperref}
\hypersetup{
            pdftitle={Meetbot: Behind the scenes},
            pdfauthor={Ricard Solé Casas},
            pdfborder={0 0 0},
            breaklinks=true}
\urlstyle{same}  % don't use monospace font for urls
\usepackage[margin=1in]{geometry}
\ifnum 0\ifxetex 1\fi\ifluatex 1\fi=0 % if pdftex
  \usepackage[shorthands=off,main=english]{babel}
\else
  \usepackage{polyglossia}
  \setmainlanguage[]{english}
\fi
% Make links footnotes instead of hotlinks:
\renewcommand{\href}[2]{#2\footnote{\url{#1}}}
\IfFileExists{parskip.sty}{%
\usepackage{parskip}
}{% else
\setlength{\parindent}{0pt}
\setlength{\parskip}{6pt plus 2pt minus 1pt}
}
\setlength{\emergencystretch}{3em}  % prevent overfull lines
\providecommand{\tightlist}{%
  \setlength{\itemsep}{0pt}\setlength{\parskip}{0pt}}
\setcounter{secnumdepth}{5}

% set default figure placement to htbp
\makeatletter
\def\fps@figure{htbp}
\makeatother

\usepackage{minted}
\usemintedstyle{autumn}

\usepackage{fontspec}
\setmonofont{Hasklig}
\defaultfontfeatures{Mapping=tex-text,Scale=MatchLowercase,Ligatures=TeX}

\title{Meetbot: Behind the scenes}
\author{Ricard Solé Casas}
\providecommand{\institute}[1]{}
\institute{Google UK \and Ada National College for Digital Skills}
\date{\today}

\begin{document}
\maketitle

\vspace*{\fill}

\section*{Declaration}

I confirm that the submitted coursework is my own work and that all
material attributed to others (whether published or unpublished) has
been clearly identified and fully acknowledged and referred to original
sources. I agree that the College has the right to submit my work to the
plagiarism detection service. TurnitinUK for originality checks.

\section*{Acknowledgements}

I'd like to thank my partner Shannon for her continued support and
challenges that help me grow, both professionally and personally. I
would also like to thank all of you who also helped me get here.

\vspace*{\fill}

{
\setcounter{tocdepth}{2}
\tableofcontents
}
\chapter{Executive Summary}\label{executive-summary}

This report provides a review and analysis of the SDLC project which
took place June 1st-2nd, 19th-21st of 2017 at Ada National College for
Digital Skills. It covers the analysis of the problem, the design
process, implementation and end product, testing and validation
techniques, and an evaluation of the project.

The task revolves around creating an automated chat bot for the Slack
platform. This discusses why the platform of choice is Slack, the
reasoning behind the technology stack choices, and how the team
self-directs itself.

We achieved:

\begin{itemize}
\tightlist
\item
  A working bot which proves our model.
\item
  Functioning team with a clear distribution of responsibilities.
\item
  A collection of suggested improvement to the bot's interface.
\end{itemize}

\chapter{Introduction}\label{introduction}

The goal of this project was to put in practice the concepts and ideas
covered in the SDLC module. A brief summary includes, but is not limited
to, the following ideas:

\begin{itemize}
\tightlist
\item
  \href{https://www.wikiwand.com/en/User_Research}{User Research}
\item
  \href{https://www.wikiwand.com/en/Agile_software_development}{Agile
  Methodologies}
\item
  \href{http://theleanstartup.com/}{Lean Startup}
\item
  \href{https://www.wikiwand.com/en/User_story}{User stories}
\end{itemize}

All of these concepts are very much intertwined and do not work
exclusively of each other. They are however, complementary to each other
and important on their own right.

\chapter{Analysis}\label{analysis}

Some of the organizers of the session encouraged the different teams to
come up with solutions to problems they were facing in their everyday
life. In our case there was a problem we encountered consistently with
our colleagues and classmates. We had a hard time organizing activities
with co-workers outside of work hours. Major issues encountered
included: platform fragmentation, hard to track plans, attendees and
their updates, no interface designed to generate plans in common
environments.

Given our experience working we decided a good MVP to prove our idea
would be a simple bot built on the already successful Slack model.

\chapter{Design}\label{design}

The first part of our design process was to define who our initial users
would be, everything would derive from that. Our decision was to focus
on \emph{professionals who use Slack for communication at work}. It is
important to highlight that we did consider other platforms, like
HipChat and Telegram or Facebook's Messenger, and there is little
stopping us from extending our bot to support multiple platforms. The
main reason driving our choice of Slack as a pioneer platform was the
ease of access to an already existing large user base on our end.

After conducting surveys, creating two personas and their corresponding
user journeys, and going through user testing on mocked bots, we
distilled our ideas and dreams into a small subset of features:

\begin{itemize}
\tightlist
\item
  Creating and deciding on a plan
\item
  Being able to list all existing plans
\item
  Update plans as they change
\end{itemize}

\chapter{Implementation}\label{implementation}

When we set out to accomplish a successful implementation to achieve
maximum market value in the minimum time to market we were faced with a
few decisions in how to implement our project. What is the goal of our
initial users? Are we helping them and, by extension, their world if we
fulfill our mission? What is our mission? What does a user's journey
look like?

After thorough discussion and analysis we decided on the core features
that would better the lives of our users: creating a plan, knowing what
plans are available, and, last but not least, joining an existing event.
These happened to be the minimal steps for our users to achieve their
goal of sharing their happy hour plans with their coworkers.

What came out of this process was a functioning bot on the Slack
platform which responded to real commands. UX sessions can be found in
our Google Drive folder:
\href{https://drive.google.com/file/d/0B6d4i1nhOpwKX19TYWFGTmd2V2M/view?usp=sharing}{Joe},
\href{https://drive.google.com/file/d/0B6d4i1nhOpwKbGVaMVlnYnY1ZWM/view?usp=sharing}{Matt},
and
\href{https://drive.google.com/file/d/0B6d4i1nhOpwKUGc0V25lcFVhSjQ/view?usp=sharing}{Danny}.

We built a bot which responded to some basic commands that allowed users
to meet IRL. Full code implementation can be found on
\href{https://github.com/ada-lunchbot/lunchbot}{Github}.

\chapter{Testing}\label{testing}

In this sequence of sessions we had two major chunks. In the first one
we tried to validate our idea without a working implementation in code.
The second consisted of taking all the lessons learned from the first
session and have a full working bot that implemented the basic needs of
our users.

In our first attempt at proving our idea a member of the team simply
changed their Slack profile to mimick that of a bot. They named
themselves \texttt{lunchbot}, changed their picture and statement to
something more \emph{bot-like}, and stopped participating in
conversations unless summoned. That helped us gather feedback on the
interface we thought we would implement to interact with the bot, along
with collecting additional features that the test users considered a
\emph{nice-to-have}.

After reflecting on the feedback from the first session we implemented a
bot and tested that with more precision. In the second round we recorded
people's screens as we asked them to complete a journey. We noticed our
concept of ``technically proficient'', and what that meant for
interacting with our user, needed to be refined. For the recorded UX
sessions please refer to links:
\href{https://drive.google.com/file/d/0B6d4i1nhOpwKX19TYWFGTmd2V2M/view?usp=sharing}{here},
\href{https://drive.google.com/file/d/0B6d4i1nhOpwKbGVaMVlnYnY1ZWM/view?usp=sharing}{here}
and
\href{https://drive.google.com/file/d/0B6d4i1nhOpwKUGc0V25lcFVhSjQ/view?usp=sharing}{here}.

\chapter{Evaluation}\label{evaluation}

If I had to name one thing to take away from this experience, it could
be distilled in the following sentence:

\begin{quote}
Test early, and test often.
\end{quote}

That is something we would've benefited from doing. We failed to test
some assumptions in the beginning and it came back to haunt us in our
sessions. Had this been a real commercial project with real
stakeholders, the fact that our bot's discoverability and help command
were unintuitive to our users would've most likely delayed launch, and
cost us a lot of money in the process.

I also benefited from learning about user journeys and how to map them.
That was a truly enlightening experience, and it made coming up with
user stories a much more trivial task.

In terms of the coding itself, I think we did a great job at
distributing the ownership of the codebase. We achieved that by having
sessions of what is commonly referred to as
\href{https://www.wikiwand.com/en/Mob_programming}{mob programming}.

\end{document}
